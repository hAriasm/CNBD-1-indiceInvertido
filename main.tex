% Template:     Informe LaTeX
% Documento:    Archivo principal
% Versión:      7.2.3 (15/06/2021)
% Codificación: UTF-8
%
% Autor: Pablo Pizarro R.
%        Facultad de Ciencias Físicas y Matemáticas
%        Universidad de Chile
%        pablo@ppizarror.com
%
% Manual template: [https://latex.ppizarror.com/informe]
% Licencia MIT:    [https://opensource.org/licenses/MIT]

% CREACIÓN DEL DOCUMENTO
\documentclass[letterpaper,oneside]{article}

% INFORMACIÓN DEL DOCUMENTO
\def\titulodelinforme {Informe de caso de estudio Nro.  01}
\def\temaatratar {Amazon Inc.}

\def\autordeldocumento {Nombre del autor}
\def\nombredelcurso {Computación en la Nube y Big Data}
\def\codigodelcurso {CO}

\def\nombreuniversidad {Universidad Nacional de San Agustín}
\def\nombrefacultad {Escuela de Posgrado}
\def\departamentouniversidad {Maestría en Ciencia de la Computación}
\def\imagendepartamento {departamentos/logo_unsa}
\def\imagendepartamentoparams {height=1.57cm}
\def\localizacionuniversidad {Arequipa, Perú}

% INTEGRANTES, PROFESORES Y FECHAS
\def\tablaintegrantes {
\begin{tabular}{ll}
	Integrantes:
	& \begin{tabular}[t]{l}
		Fredy Abel Huanca Torres \\
		Henrry Arias \\
		Edison Pérez
	\end{tabular} \\
	Profesor:
	& \begin{tabular}[t]{l}
		Alvaro Henrry Mamani Aliaga
	\end{tabular} \\
	\multicolumn{2}{l}{Fecha de realización: \today} \\
	\multicolumn{2}{l}{Fecha de entrega: \today} \\
	\multicolumn{2}{l}{\localizacionuniversidad}
\end{tabular}}{
}

% IMPORTACIÓN DEL TEMPLATE
\input{template}

% INICIO DE PÁGINAS
\begin{document}
	
% PORTADA
\templatePortrait

% CONFIGURACIÓN DE PÁGINA Y ENCABEZADOS
\templatePagecfg

% RESUMEN O ABSTRACT
\begin{resumen}
	El presente, es un informe sobre un caso de estudio: Amazon inc. el mismo que ha usado el framework Hadoop para el procesamiento y almacenamiento de los datos que generan, son soportados por artículos de investigación y datos obtenidos de recursos de internet, el cual ha servidor para realizar este informe, asimismo consideramos el volumen, variedad, velocidad y veracidad de los datos que procesa.
\end{resumen}

% TABLA DE CONTENIDOS - ÍNDICE
\templateIndex

% CONFIGURACIONES FINALES
\templateFinalcfg

% ======================= INICIO DEL DOCUMENTO =======================

% Template:     Informe LaTeX
% Documento:    Archivo de ejemplo
% Versión:      7.2.3 (15/06/2021)
% Codificación: UTF-8
%
% Autor: Pablo Pizarro R.
%        Facultad de Ciencias Físicas y Matemáticas
%        Universidad de Chile
%        pablo@ppizarror.com
%
% Manual template: [https://latex.ppizarror.com/informe]
% Licencia MIT:    [https://opensource.org/licenses/MIT]

% ------------------------------------------------------------------------------
% NUEVA SECCIÓN
% ------------------------------------------------------------------------------
% Las secciones se inician con \section, si se quiere una sección sin número se
% pueden usar las funciones \sectionanum (sección sin número) o la función
% \sectionanumnoi para crear el mismo título sin numerar y sin aparecer en el índice
\section{Cuál es su necesidad}

\lipsum[2] % Párrafo ejemplo, se puede borrar	

% ------------------------------------------------------------------------------
% NUEVA SECCIÓN
% ------------------------------------------------------------------------------
\clearpage
\section{Cómo lo utiliza.}

\lipsum[2] % Párrafo ejemplo, se puede borrar	


% ------------------------------------------------------------------------------
% NUEVA SECCIÓN
% ------------------------------------------------------------------------------
\clearpage
\section{Volumen}

\lipsum[1] % Párrafo ejemplo, se puede borrar	

\subsection{Cuál es el volumen de datos que almacena}
\lipsum[1] % Párrafo ejemplo, se puede borrar	

\subsection{Cuál es el volumen de datos que procesa}
\lipsum[1] % Párrafo ejemplo, se puede borrar	


% ------------------------------------------------------------------------------
% NUEVA SECCIÓN
% ------------------------------------------------------------------------------
\clearpage
\section{Variedad}
Cómo son sus datos
\lipsum[1] % Párrafo ejemplo, se puede borrar	

% ------------------------------------------------------------------------------
% NUEVA SECCIÓN
% ------------------------------------------------------------------------------
\clearpage
\section{Velocidad}

\subsection{Cuanto es el tiempo en que procesaba antes de usar Hadoop}
\lipsum[1] % Párrafo ejemplo, se puede borrar	

\subsection{Cuanto es el tiempo que procesa ahora que usa Hadoop}

% ------------------------------------------------------------------------------
% NUEVA SECCIÓN
% ------------------------------------------------------------------------------
\clearpage
\section{Veracidad}

\subsection{Como probar que los datos que procesa sean válidos}
\lipsum[1] % Párrafo ejemplo, se puede borrar	


% ------------------------------------------------------------------------------
% REFERENCIAS (ESTILO BIBTEX), revisar configuración \stylecitereferences
% ------------------------------------------------------------------------------
\clearpage % Salto de página
\begin{references}
	\bibitem{templateinforme}
	Template Informe en \LaTeX.
	\textit{¡Revisa el manual online de este template!} \\
	\url{https://latex.ppizarror.com/informe}

	\bibitem{excel2latex}
	Excel2Latex.
	\textit{Importa de forma sencilla tus tablas de Excel a \LaTeX.} \\
	\url{https://www.ctan.org/tex-archive/support/excel2latex/}

	\bibitem{overleaf}
	Overleaf.
	\textit{Uno de los mejores editores online para \LaTeX, renovado con su versión 2.0.} \\
	\href{https://www.overleaf.com?r=298b935f&rm=d&rs=b}{\hreftext{https://es.overleaf.com/}}
	
	\bibitem{tablesgenerator}
	Tables Generator.
	\textit{Creador de tablas online para \LaTeX.}\\
	\url{https://www.tablesgenerator.com}
\end{references} % Ejemplo, se puede borrar

% FIN DEL DOCUMENTO
\end{document}
