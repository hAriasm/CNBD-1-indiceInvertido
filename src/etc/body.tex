% Template:     Informe LaTeX
% Documento:    Archivo de ejemplo
% Versión:      8.1.7 (24/07/2022)
% Codificación: UTF-8
%
% Autor: Pablo Pizarro R.
%        pablo@ppizarror.com
%
% Manual template: [https://latex.ppizarror.com/informe]
% Licencia MIT:    [https://opensource.org/licenses/MIT]

% ------------------------------------------------------------------------------
% NUEVA SECCIÓN
% ------------------------------------------------------------------------------
% Las secciones se inician con \section, si se quiere una sección sin número se
% pueden usar las funciones \sectionanum (sección sin número) o la función
% \sectionanumnoi para crear el mismo título sin numerar y sin aparecer en el índice
\section{Cuál es su necesidad}

\lipsum[2] % Párrafo ejemplo, se puede borrar	


% ------------------------------------------------------------------------------
% NUEVA SECCIÓN
% ------------------------------------------------------------------------------
\clearpage
\section{Cómo lo utiliza}

\lipsum[2] % Párrafo ejemplo, se puede borrar	

% ------------------------------------------------------------------------------
% NUEVA SECCIÓN
% ------------------------------------------------------------------------------
% Inserta una sección sin número
\clearpage
\section{Volumen}

\lipsum[1] % Párrafo ejemplo, se puede borrar	

\subsection{Cuál es el volumen de datos que almacena}
\lipsum[1] % Párrafo ejemplo, se puede borrar	

\subsection{Cuál es el volumen de datos que procesa}
\lipsum[1] % Párrafo ejemplo, se puede borrar	

% ------------------------------------------------------------------------------
% NUEVA SECCIÓN
% ------------------------------------------------------------------------------
\clearpage
\section{Variedad}
Cómo son sus datos
\lipsum[1] % Párrafo ejemplo, se puede borrar	

% ------------------------------------------------------------------------------
% NUEVA SECCIӓN
% ------------------------------------------------------------------------------
\clearpage
\section{Velocidad}

\subsection{Cuanto es el tiempo en que procesaba antes de usar Hadoop}
\lipsum[1]\scite{overleaf} % Párrafo ejemplo, se puede borrar	

\subsection{Cuanto es el tiempo que procesa ahora que usa Hadoop}


% ------------------------------------------------------------------------------
% NUEVA SECCIӓN
% ------------------------------------------------------------------------------
\clearpage
\section{Veracidad}

\subsection{Como probar que los datos que procesa sean válidos}
\lipsum[1] % Párrafo ejemplo, se puede borrar	

% ------------------------------------------------------------------------------
% REFERENCIAS, revisar configuración \stylecitereferences
% ------------------------------------------------------------------------------
\clearpage
\bibliography{library}